\documentclass[a4paper, 12pt]{article}
\usepackage{amsfonts,amsmath,amsthm}
\usepackage{graphicx}
\usepackage{authblk}
\usepackage{hyperref}
\usepackage{microtype}
\usepackage{lipsum}
\usepackage[section]{placeins}
% \usepackage{pdflscape}
\usepackage{afterpage}
\usepackage{capt-of}% or use the larger `caption` package
\usepackage{pifont}% http://ctan.org/pkg/pifont
\usepackage{bm}
\usepackage{pdfpages}
\usepackage{mdframed}
\usepackage{makecell}
\usepackage{hyperref}
\usepackage{todonotes}
\usepackage{xcolor}

\DeclareMathOperator*{\argmax}{arg\,max}
\DeclareMathOperator*{\argmin}{arg\,min}
\newcommand{\cmark}{\ding{51}}%
\newcommand{\xmark}{\ding{55}}%

\newtheorem{thm}{Theorem}
\newtheorem{lem}[thm]{Lemma}
\newtheorem{prp}[thm]{Proposition}
\newtheorem{rem}[thm]{Remarks}
\newtheorem{cor}[thm]{Corollary}
\newtheorem{dfn}[thm]{Definition}
\newtheorem{prt}[thm]{Property}
\newtheorem{algo}{Algorithm}

%\def\Dd{$\Delta^d$}
\def\vol{\mbox{vol}}
\def\NN{{\mathbb N}}
\def\RR{{\mathbb R}}
\def \d{{\mathrm d}}
\def \e{{\mathrm e}}
\def \c++{{\tt C++}}
\def\volesti{{\tt volesti}}
\def\cran{{\tt CRAN}}
\def\R{{\tt R}}

\usepackage{geometry}
\geometry{margin=0.7in}

%\usepackage{biblatex}
%\addbibresource{biblio.bib}

\begin{document}

\begin{center}
    \LARGE{Personal Qualifications}
\end{center}

\section{Introduction}

My name is Haris Zafeiropoulos and I am a PhD student in the Biology Department at University of Crete. My research focuses on deciphering the underlying mechanisms of ecosystem functioning at the microbial level via bioinformatics and computational methods. 

%My name is Apostolos Chalkis. I am a PhD student in Computer Science at the University of Athens. My research focuses on geometric random walks for high dimensional sampling with applications in volume computation, optimization and financial modeling.

\section{Background}
After my Bachelor in Biology, I continued my studies in Computer Science having my Master in Bioinformatics. It was then when I started coding in terms of an open source programmer and I developed \textcolor{blue}{\href
{https://github.com/hariszaf/pema}{PEMA}}; a pipeline for the support of metabarcoding analysis, which has been selected from the \textcolor{blue}{\href{https://www.lifewatch.eu/}{LifeWatch ERIC}} to be included in its e-Infrastructure. Furthermore, during the first year of my PhD, I concluded a \textcolor{blue}{\href{http://dnaqua.net/wp-content/uploads/2019/08/Zafeiropoulos.pdf}{Short Term Scientific Mission}} of the \textcolor{blue}{\href{https://dnaqua.net/stsms/}{DNAqua-net}} \textcolor{blue}{\href{https://www.cost.eu/}{COST}} action.

In the framework of my PhD, I started working with biological processes, especially ones related with metabolism. Thus, the analysis of metabolic networks was the next step for my research. Combining Biology with a solid understanding on Mathematical topics, I was able to join the \textcolor{blue}{\href{https://geomscale.github.io/}{GeomScale group}} in the framework of \textcolor{blue}{\href{https://summerofcode.withgoogle.com/archive/2020/organizations/4606415412396032/}{Google Summer of Code (GSoC) 2020}} as a Mentor. It is this combination along with the well established collaboration with the GeomScale group that will allow me to complete this project, addressing both its biological and computational challenges. 

%My bachelor and master studies was focused on computational mathematics and statistics, and algorithms. The main field of my PhD is computational geometry, statistics and mathematical software.
%Thus, I have a very good understanding and solid theoretical background on the topic of the current proposal as I have been doing research and contribute with some new results.

%Moreover, I am an active open source programmer. 
%I have participated in Google summer of Code (GSoC) 2018 and 2019 as a student / intern and I am one of the authors of package \volesti. 
%During GSoC 2018 I extended \volesti\ with new efficient sampling and volume approximation algorithms which lead to the first {\tt CRAN} version. 
%During GSoC 2019 I equipped \volesti\ with Hamiltonian Monte Carlo samplers for uniform and spherical Gaussian sampling and new volume approximation algorithms. I am currently, one of the administrators and mentors of {\tt GeomScale} organization in GSoC. %Hence, I know very well the structure of \volesti\ and the steps I have to implement to complete successfully this project.

\section{Trajectory}
I am very excited with the opportunity of Tweag Fellowship because I will be able to combine my PhD studies with a fellowship of great quality;
without taking an unpaid leave. It is my belief that this project is of high value both for the community overall, especially for the Human Physiology research field, but also for my personal research interests as it will be a fundamental step for inferring/ microbial interactions via networks of metabolic networks. Having said that, the {\tt Tweag} Fellowship will be a decisive contribution to my PhD studies and make the next, rather novel, step of my scientific journey.


%the practical study of mixing time, besides its practical importance will lead to a high quality paper submission in a statistical conference. Moreover, it is a natural extension to study metabolic networks of thousands of reactions and metabolites for the first time and hence provide options to biologists that were not possible in the past. 
{%\tt Tweag} Fellowship will be a decisive contribution to my PhD studies to help me produce high quality research results and make the next step of my scientific journey.
%\todo[inline]{and you will also contribute to the company, to the open source community and to the world.}

\section{Working Conditions}
I would like to start the project on 4 of January 2021 and end on 26 March 2021 (12 weeks), but of course I am open to other suggestions. I will work remotely from Athens, Greece. I will not have any other projects or source of income during that period. This project will be a full time job for me. I will not need any additional resource (hardware, office, relocation etc).
%\begin{itemize}
%    \item \textbf{Starting date:} 1st September 2020. \textbf{Ending date:} 24 November 2020 (12 weeks).
%    \item I will work remotely from Athens, Greece. I will not have any other projects or source of income during %that period. This project will be a full time job for me.
%    \item I will not need any additional resource (hardware, office, relocation etc).
%\end{itemize}


\section{References}
\begin{enumerate}
    \item Dr. \textcolor{blue}{\href{https://vissarion.github.io/}{Vissarion Fisikopoulos}} (mentor). Software Engineer @ Oracle, Research Scientist @ University of Athens. He has long experience in open source software development.
    \item Dr. \textcolor{blue}{\href{https://who.paris.inria.fr/Elias.Tsigaridas/}{Elias Tsigaridas}} (mentor). Research scientist at INRIA Paris. He is an expert on algebraic algorithms and mathematical software.
    \item \textcolor{blue}{\href{https://tolischal.github.io/aboutme/}{Apostolos Chalkis}} (mentor). PhD student in Computer Science at the Department of Telecomuncations and Informatics in University of Athens.  
\end{enumerate}
%\section{Other Details}

%\printbibliography[title=Bibliography]


\end{document}

